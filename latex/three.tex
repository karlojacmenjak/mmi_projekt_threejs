\documentclass[a4paper,12pt]{article}

\usepackage[a4paper,left=2.54cm,right=2.54cm,top=2.54cm,bottom=2.54cm]{geometry}

\usepackage[croatian]{babel}
\usepackage[utf8]{inputenc}
\usepackage{times}

\usepackage{amsmath} 
\usepackage{amssymb}

\usepackage{setspace}
\onehalfspacing

\usepackage{titlesec}
\titleformat{\section}{\fontsize{16pt}{20pt}\selectfont\bfseries}{\thesection.}{0.4cm}{}
\titleformat{\subsection}{\fontsize{14pt}{18pt}\selectfont\bfseries}{\thesubsection.}{0.4cm}{}
\setlength{\parskip}{10pt}

\titlespacing*{\section}{0pt}{0.5cm}{0pt}
\titlespacing*{\subsection}{0pt}{0.5cm}{0pt}

\usepackage{enumitem}
\setlist{topsep=3pt,itemsep=3pt}

\usepackage{graphicx}

\usepackage[numbers]{natbib}
\setlength{\bibsep}{2pt}


\begin{document}

%%%%%%%%%%%%%%%%%%%%%%%%%%% NASLOVNICA %%%%%%%%%%%%%%%%%%%%%%%%%%%%%%%%%%%%%%%%%%%%%%%%%%%
\thispagestyle{empty}
\begin{center}
Sveučilište u Zagrebu\\
Fakultet organizacije i informatike
\end{center}
\vfill
\begin{center}
\Large Uvod u THREEjs
\end{center}
\vfill
U Varaždinu, 2.1.2023.\hfill Tim: Karlo Jačmenjak, Antonio Kupčić, Josip Mojzeš


\newpage
\setcounter{page}{1}
%%%%%%%%%%%%%%%%%%%%%% KRAJ NASLOVNICE %%%%%%%%%%%%%%%%%%%%%%%%%%%%%%%%%%%%%%%%%%%%%%%%%%%

\section{Uvod}
\textbf{Kardioida} je geometrijsko mjesto odabrane točke $T$ na kružnici koja se kotrlja (bez klizanja) po drugoj fiksnoj kružnici istog polumjera. Na temelju definicije kardioide mogu se izvesti njezine parametarske jednadžbe. Parametarske jednadžbe nisu jedinstvene jer ovise o tome gdje ćemo u ravnini postaviti fiksnu kružnicu i koju ćemo točku odabrati na kružnici koja se kotrlja.\par 
Ukoliko fiksnu kružnicu polumjera \(a\) postavimo tako da ima središte \(S_1\) u ishodištu koordinatnog sustava, nadalje ako na početku kružnicu koja se kotrlja postavimo tako da ima središte $S_2$ u točki $(2a,0)$, a na njoj odaberemo točku $T$ s koordinatama $(a,0)$, tada parametarske jednadžbe kardioide glase
\begin{align*}
x&=a\big(2\cos{t}-\cos{2t})\\ 
y&=a\big(2\sin{t}-\sin{2t}\big).
\end{align*}
pri čemu je $t\in[0,2\pi]$. U donjem apletu je upravo takva početna situacija (kada je $t=0$) 
i stavljeno je da kružnice imaju polumjer $1$, tj. $a=1$.\par 
O parametru $t$ možemo razmišljati kao o vremenu. U promatranom trenutku $t$ kružnica koja se kotrlja se nalazi u nekom određenom položaju 
s obzirom na fiksnu kružnicu, a samim time i odabrana točka $T$ se nalazi na određenoj poziciji u ravnini.
\subsection{Podnaslov unutar uvoda}
Na primjer, definicija derivacije funkcije $f:I\to\mathbb{R}$ u točki $x_0\in I$ glasi
$$f'(x_0)=\lim_{\Delta x\to 0}{\frac{f(x_0+\Delta x)-f(x_0)}{\Delta x}}.$$
Ako želimo formulu automatski numerirati,
\begin{equation}
f'(x_0)=\lim_{\Delta x\to 0}{\frac{f(x_0+\Delta x)-f(x_0)}{\Delta x}},
\end{equation}
ili ju želimo označiti svojim simbolom
\begin{equation}
f'(x_0)=\lim_{\Delta x\to 0}{\frac{f(x_0+\Delta x)-f(x_0)}{\Delta x}}.\tag{$\clubsuit$}
\end{equation}

\newpage

\section{Aplet}
Cilj apleta je vizualno ilustrirati definiciju kardioide. Opišimo ukratko funkcioniranje apleta:
\begin{itemize}
\item Jedino što možete mijenjati u apletu je vrijednost parametra $t$ pomoću miša.
\item Za preciznije i sporije kretanje točke $T$, parametar $t$ mijenjajte pomoću strelica na tastaturi 
tako da najprije mišem kliknete na kružić od slidera, a nakon toga strelicama lijevo-desno mijenjate vrijednosti parametra $t$.
\item Pritiskom na tipku s trokutićem u donjem lijevom kutu možete pokrenuti animaciju tako da se parametar $t$ sam mijenja. Animaciju možete prekinuti pritiskom na tu istu tipku.
\item Prilikom kotrljanja kružnice točka $T$ ostavlja trag tako da se jasno vidi njezino geometrijsko mjesto točaka koje zovemo kardioida.
\item Ukoliko aplet ima fokus, pritiskom na \verb|CTRL+F| možete obrisati trag koji je ostavila točka $T$ prilikom kotrljanja kružnice.
\item Pritiskom na tipku u gornjem desnom kutu možete odmah vratiti aplet na početno zadane uvjete.
\end{itemize}
\LaTeX{} može ubaciti vanjsku sliku u svoj dokument. Slika pritom mora biti u odgovarajućem formatu i najjednostavnije je da se nalazi u tekućem
direktoriju \verb|tex| datoteke. Nadalje, \LaTeX{} ima dosta svojih fantastičnih paketa za crtanje slika kao što je \verb|tikz| paket.\par
\vspace*{5mm}
\begin{figure}[!h]
\centering

\caption{Kardioida u \texttt{GeoGebri}}
\end{figure}

\paragraph{Referenciranje na literaturu.} Prema literaturi \cite{Maric} vrijedi\,\ldots \ Prema literaturi \cite{geo} mora biti\,\ldots

\begin{thebibliography}{9}
\bibitem{Maric} An\dj elko Mari\'c, \emph{Vektori -- zbirka rije\v{s}enih zadataka}, Element, Zagreb, 1997.
\bibitem{geo} GeoGebra, \texttt{http://www.geogebra.org/cms/}, (9.3.2014.)
\end{thebibliography}

\end{document}